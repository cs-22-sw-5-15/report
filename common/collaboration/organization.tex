\subsection{The ECDAR-SW5 Organization}\label{sub:ecdar-organization}
To better orchestrate collaboration between all Scrum groups, it is essential to organize how communication will take place, as well as how we will distribute and evaluate work.
%There must be some kind of organization in order to orchestrate collaboration between all the project groups that are working on the ECDAR project. 
%Organisation is important because it establishes channels of communication, a distribution of the workload, and a process for evaluating the outcome of each stage in the development process.

Each of the project groups are themselves a Scrum team.
As Scrum is dependent on the teams being small, the organization must use an agile scaling framework.
Such a framework ensures distribution of the workload between the teams and that the output from each team can be integrated. 


As an organization, the ECDAR-SW5 organization found it crucial to consider its needs before picking a framework to implement.
Due to the requirements from Aalborg University as well as what is preferred within the six individual groups, each group must be self-managed and free to decide how they work internally.
From these requirements we decided that Scrum of Scrums was a good way of scaling our organization. Scrum of Scrums was chosen, as Scrum was already known, and preferred. The Scrum of Scrums framework allowed each team to still work as one team, while still providing teamwork across teams. Other possible solutions exist, we might have chosen to use eXtreme Programming, or any other agile framework. The choice of Scrum of Scrums, was made due to it being the most well known, and well understood method.
%This led us to choose to work in "Scrum of Scrums", allowing each group to work in their own way without sacrificing the teamwork that Scrum provides across the organization.

%As agile is very popular within some of the largest organizations in the world(SOURCE PLEASE), multiple different agile scaling frameworks have been invented. 
