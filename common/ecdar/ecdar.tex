\section{Introduction to ECDAR}\label{sec:introduction-to-ecdar}
\textit{The following section and its subsections are written in collaboration with other ECDAR project groups.}\vspace{1em}

This section serves as an overview over the different concepts of ECDAR to provide a basic understanding of what ECDAR is, the purpose of ECDAR, its architecture, the technologies behind it, and what state ECDAR is currently in.

ECDAR stands for \textbf{E}nvironment for \textbf{C}ompositional \textbf{D}esign and \textbf{A}nalysis of \textbf{R}eal Time Systems.
ECDAR is a graphical tool to model real-time systems using timed input/output automata, and analyse these systems. 
This process is called model checking. 

%what
Model checking is a method of verifying that system models in the form of finite-state automata are correct \cite{modelchecking-handbook}. 
The bigger a model becomes, the harder it becomes to verify each edge case. 
Having computer-aided checking ensures that all edge cases are checked, where one can be forgotten if done by hand.

%why
What ECDAR is useful for is best described by an example \label{ECDAR:satellite}:

Imagine you are working for a space program and want to send a satellite into space.
You spent several months testing the system to make sure the model is correct and everything is working as intended.
The day after launch, you find a major issue in the system.
The satellite is already in space, and you cannot simply just update it, like you usually would with other systems. 
This is why ECDAR is a useful tool to model check systems, to make sure that a system's components works together as expected.

Model checking is a way of checking if a model is true to a given specification.
This is usually associated with hardware or systems that have a liveness, since validity of the life cycle needs checking.
There is also a need for certain safety requirements to ensure that there are no crashes or other obstacles throughout the life cycle.

It is important to model check a system to ensure that it behaves as intended.
In the above example with the faulty satellite, there are several, interdependent components.
If one of these components fail or are blocked, then it makes the entire satellite faulty.
% Take for instance the satellite example, it is a very complex hardware/system with a lot of functionality.
% The satellite itself has several components, which are dependent on each other.

By providing a tool that can both be reliable and productive at testing if a new system is true to a given specification, we can not only ensure correctness but also efficiency.

It does need to be noted that there still are some issues with using ECDAR as a tool for model checking.
ECDAR can try to guide the user as much as possible to make sure the specification is valid, but in the end, the user can still make a wrong specification.
%Also how efficient ECDAR is as a tool is dependent on what kind of algorithms are used, and the efficiency of the run-time.

%%ECDAR therefore tests the correctness of the model using timed I/O automata.
%% It parallelizes test-case generation and test execution to provide this significant speed-up.

To sum up, what the purpose of ECDAR is:
\begin{quote}
"[to integrate] conformance testing into a new IDE that now features
modelling, verification, and testing. The new tool uses model-based mutation testing, requiring only
the model and the system under test, to locate faults and to prove the absence of certain types of faults." \cite{Gundersen_2018}
\end{quote}



