\section{Sprint One}
The first sprint was spent getting a better understanding of the problem, the requirements, the theory, and the codebase. Beyond that, the Scrum Master from our team had to write, with the other Scrum Masters, some common sections about Ecdar and the work-method between groups, meaning the first sprint was spent doing quite a bit of research.

\subsection{Plans}
To get a better understanding of the problem a few things had to be researched:
\begin{itemize}
    \item Organization of the work inside our group as well as between the other groups.
    \item When the feature could be considered done.
    \item If there were any parts of the theory we should keep in mind.
    \item The different scenarios when a clock is redundant.
    \item Where in the codebase the feature should be implemented.
\end{itemize}

\subsection{Done criteria}
Our product owner gave us a definition of done which consists of
\begin{itemize}
    \item Our methods should remove some clocks in cases that we specify ourselves
    \item All refinement/consistency check tests pass successfully
    \item We can convincingly reason that our rules theoretically does not change the behavior in all cases. We can optionally formally prove it.
\end{itemize}

The done criteria are the main goals in this project, where the sprints are used to solve to-dos in the product backlog which is targeted at the criteria. The done criteria should therefore act as a guide for the scrum team where sprints are planned with the done criteria in mind. Each sprint review the scrum team will look at the done criteria to make sure they are on the right path, and how far they are from meeting the criteria.

\subsection{Relevant Theory}
While we are not creating additional theory, it is crucial to understand the relevant parts of the theory, and how to go about it. The primary theory we need to be aware of are DBMs (Section \ref{sec:DBM}), where we need to understand both how they are implemented, and what effect removing a clock has on them.

Theory-wise, when removing a clock, the corresponding row and column needs to be removed from the DBM. Luckily, the implementation is fairly straight forward. It is structured in a way that we only need to modify a hashmap with the clock definitions along with the guards they are used within and of course modifying the federation by either removing the clock, or ignore the clock by setting all of its constraints to either $\infty$ or $0$ as it fits.

\subsection{Scenarios} \label{scenarios}
When trying to reduce clocks there are multiple scenarios which can be looked into. One scenario is a clock being initiated but is not involved in constraints. The clock can in this situation be removed, as it does not violate any constraints. 

Another scenario is when components are combined using conjunction and parallel composition, one clock can affect the importance of the other resulting in the first clock becoming redundant. This can be done if the constraints becomes redundant when combining the clocks.

Another scenario where clocks from two components can be combined is if the two clocks both reset at the same time. Every situation needs to be analyzed to make sure that they reset at the same time, every time. Suppose in figure \ref{fig:tioa-scenario1} that the edges before and after the clock reset in each component only had output actions and no guards. If the two components were to be combined into one, the clocks could also be combined.  % they both only resets upon the same output action.


\begin{figure}[H]
    \centering
    \begin{tikzpicture}
        \node[state] (as) {$a_s$};
        \node[state, right=1cm of as] (1) {$a_1$};
        \node[state, right=2cm of 1] (2) {$a_2$};
        \node[state, right=1cm of 2] (af) {$a_f$};
        \node[state, below=.5cm of as] (bs) {$b_s$};
        \node[state, right=1cm of bs] (3) {$b_1$};
        \node[state, right=2cm of 3] (4) {$b_2$};
        \node[state, right=1cm of 4] (bf) {$b_f$};
        \draw   (1) edge[above, arrow] node{$o!$\hspace{1em}$x := 0$} (2)
                (3) edge[arrow, above] node{$o!$\hspace{1em}$y := 0$} (4)
                (as) edge[arrow] node[fill=white, font=\bfseries, pos=0.4]{\Large ?} (1)
                (bs) edge[arrow] node[fill=white, font=\bfseries, pos=0.4]{\Large ?} (3)
                (2) edge[arrow] node[fill=white, font=\bfseries, pos=0.4]{\Large ?} (af)
                (4) edge[arrow] node[fill=white, font=\bfseries, pos=0.4]{\Large ?} (bf)
        ;
    \end{tikzpicture}
    \caption{Two TIOA with clocks that reset at the same time.}
    \label{fig:tioa-scenario1}
\end{figure}

However in this scenario there are a few things to look out for, where the clocks cannot be combined. First scenario being if one of the components is cyclic like shown in figure \ref{fig:tioa-scenario2}. If the TIOA is cyclic we cannot prove that the two TIOAs will reset their clock at the same time. Therefore we can not prove that the model will be entirely the same when combining the two clocks. 

\begin{figure}[H]
    \centering
    \begin{tikzpicture}
        \node[state] (as) {$a_s$};
        \node[state, right=1cm of as] (1) {$a_1$};
        \node[state, right=2cm of 1] (2) {$a_2$};
        \node[state, right=1cm of 2] (af) {$a_f$};
        \node[state, below=1cm of as] (bs) {$b_s$};
        \node[state, right=1cm of bs] (3) {$b_1$};
        \node[state, right=2cm of 3] (4) {$b_2$};
        \node[state, right=1cm of 4] (bf) {$b_f$};
        \draw   (1) edge[above, arrow] node{$o!$\hspace{1em}$x := 0$} (2)
                (3) edge[arrow, above] node{$o!$\hspace{1em}$y := 0$} (4)
                (as) edge[arrow] node[fill=white, font=\bfseries, pos=0.4]{\Large ?} (1)
                (bs) edge[arrow] node[fill=white, font=\bfseries, pos=0.4]{\Large ?} (3)
                (2) edge[arrow] node[fill=white, font=\bfseries, pos=0.4]{\Large ?} (af)
                (4) edge[arrow] node[fill=white, font=\bfseries, pos=0.4]{\Large ?} (bf)
                (1) edge[arrow, bend right=2cm] node[above, font=\bfseries]{$o!$} (as)
        ;
    \end{tikzpicture}
    \caption{Two TIOAs where the clocks cannot be combined since one of them is cyclic.}
    \label{fig:tioa-scenario2}
\end{figure}


Another thing we need to check the components for is whether one of the components have the ability to make the output action that resets the clock before the other. An example of this scenario is shown in figure \ref{fig:tioa-scenario3} where the second component has the guard $y > 3$. Here we cannot be sure that the two clocks, $x$ and $y$, will reset at the same time, since the second component is dependent on a guard in order to reset.

\begin{figure}[H]
    \centering
    \begin{tikzpicture}
        \node[state] (as) {$a_s$};
        \node[state, right=2cm of as] (1) {$a_1$};
        \node[state, right=2cm of 1] (2) {$a_2$};
        \node[state, right=1cm of 2] (af) {$a_f$};
        \node[state, below=1cm of as] (bs) {$b_s$};
        \node[state, right=2cm of bs] (3) {$b_1$};
        \node[state, right=2cm of 3] (4) {$b_2$};
        \node[state, right=1cm of 4] (bf) {$b_f$};
        \draw   (1) edge[above, arrow] node{$o!$\hspace{1em}$x := 0$} (2)
                (3) edge[arrow, above] node{$o!$\hspace{1em}$y := 0$} (4)
                (as) edge[arrow] node[font=\bfseries, above]{$o!$} (1)
                (bs) edge[arrow] node[font=\bfseries, above]{$y>3\hspace{1em}o!$} (3)
                (2) edge[arrow] node[fill=white, font=\bfseries, pos=0.4]{\Large ?} (af)
                (4) edge[arrow] node[fill=white, font=\bfseries, pos=0.4]{\Large ?} (bf)
        ;
    \end{tikzpicture}
    \caption{Two TIOAs where the clocks $x$ and $y$ cannot be combined.}
    \label{fig:tioa-scenario3}
\end{figure}



\subsection{Codebase}
In the codebase for the Reveaal engine, where the solution for this project has to be implemented, we are going to work around the components. More specifically, before a component is assigned its clocks, an analysis of the component should be performed. 

There are however a question arising when the analysis happens before components are combined. How to know if a clock will become redundant before components are combined? The analysis has to take into account that components 

\subsection{Review}
This sprint had a big focus on researching and understanding the problem. Generally, it was successful. The Scrum Master finished the common sections for the report, while the group researched where they became more familiar with the codebase, theory, and the problem in general.

Theory-wise, the primary researched part was DBMs, since this is the more relevant to our problem of reducing clocks. This problems criteria for a solution was also given by the Product Owner, along with a few scenarios for when a clock can be reduced, with ranging difficulty.

Generally, the group did not find any elements of the problem making it seem either too much or too little to work with. The problem is naturally scalable in the sense that there are multiple scenarios for clock reduction, meaning we only need to implement the scenarios that seems doable within our capabilities and time constraint.



%Vi har fået en del content, 

%I forhold til. fællesskrivning:  Ecdar, Agile og Collaboration er done.

%Der er blevet skrevet om teori og kodebasen. 

%Teorimæssigt er. der skrevet om. TIOA og DMBs. Vi vil gerne have skrevet lidt mere om dbm (måske)? Der er skrevet om federations under dbm. Der mangler at blive skrevet om model checking

%Kodebasen er der skrevet lidt om, men nok til at informere læseren om hvor præcis klok analysen skal foregå i koden.

%Folk synes generelt, at opgaven er som vi regnede med.

%Hvor nemt er det at implementere i Rust? Vi skal have bedre overblik over kodebasen.

%Det advandcerede eksempel lyder spændende, men er mere kompliceret.

\iffalse
Mangler:
Omskriv model checking. BALINGO
Skriv om transition systems. Magnus
Define what a specification is. Magnus
Beslut om der er skrevet nok til dbm. Simon
Få review fra Martijn implementeret. Kira
Review. Loppe
\fi
\subsection{Retrospective}
This was the first sprint of this project, and generally caused some confusion. We were not completely aware when the sprint had actually started, meaning the start was a little hectic and poorly planned. Beyond that, a good deal of sickness haunted the group, meaning we were rarely present at the same time, making it hard to work together and agree on workload and -tasks.

Generally, we found some points to improve upon for next sprint:
\begin{itemize}
    \item Clearer communication regarding project status and work
    \item Better planing and definition tasks - Better usage of project boards
    \item Stick better to the sprint plan
\end{itemize}
These points will be considered heavily for the next sprint.

One element from our process which worked good was our daily sprint meetings, where we were all caught up on what we were doing. This fell a bit short though because of the reasons above.





%Der har været få fremmødte.
%Opgaver var lidt uklare, men det var også research opgaver.

%Det var uklart at sprintet var igang. Vi skal være tydelige i, at vi holder os til sprint planning.

%Det er godt at holde daily scrum/ sock check.

%Vi håndterede diskussion om arbejdstider ved at holde et møde. Vi aftalte at torsdag er arbejdsdag. Andet fravær skal du selv finde ud af at gøre op for. Vi skal arbejde lidt udenfor 'arbejdstiden'. Dette vil være velafgrænsede opgaver.

%arbejdsopgaver skal være mere veldefinerede. Vi vil, næste sprint, bruge project board og evaluere til næste sprint review. 